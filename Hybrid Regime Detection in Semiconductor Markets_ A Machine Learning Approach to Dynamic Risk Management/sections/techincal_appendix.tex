\appendix
\section{Technical Appendix}

\subsection{Implementation Architecture}
The hybrid regime detection framework consists of three main components:

\subsubsection{Data Preprocessing Pipeline}
\begin{itemize}
    \item Daily return calculation using adjusted closing prices
    \item Rolling window features (21-day and 63-day):
        \begin{itemize}
            \item Volatility (standard deviation)
            \item Momentum (cumulative returns)
            \item Trading volume metrics
        \end{itemize}
    \item Standardization using rolling z-scores
\end{itemize}

\subsubsection{HMM Base Model Configuration}
\begin{itemize}
    \item Number of states: 3 (Low, Medium, High Volatility)
    \item Features: Returns, Volatility, Volume
    \item Training window: 252 trading days
    \item Initialization: k-means clustering
    \item Convergence criteria: 1e-6 tolerance
    \item Maximum iterations: 1000
\end{itemize}

\subsubsection{LSTM Enhancement Layer}
\begin{itemize}
    \item Architecture:
        \begin{itemize}
            \item Input layer: 5 features × 21 time steps
            \item LSTM layer: 64 units, dropout 0.2
            \item Dense layer: 32 units, ReLU activation
            \item Output layer: 3 units, softmax activation
        \end{itemize}
    \item Training parameters:
        \begin{itemize}
            \item Batch size: 32
            \item Epochs: 100 with early stopping
            \item Optimizer: Adam (lr=0.001)
            \item Loss function: Categorical crossentropy
        \end{itemize}
\end{itemize}

\subsection{Regime Detection Algorithm}
The regime detection process follows these steps:

\begin{enumerate}
    \item Base HMM prediction using volatility thresholds
    \item LSTM prediction on rolling window data
    \item Ensemble combination rules:
        \begin{itemize}
            \item If LSTM confidence > 0.8, use LSTM prediction
            \item If HMM and LSTM agree, use common prediction
            \item Otherwise, default to HMM prediction
        \end{itemize}
    \item Smoothing using 5-day mode filter
\end{enumerate}

\subsection{Risk Metrics Calculation}
\begin{itemize}
    \item Value at Risk (VaR):
        \begin{itemize}
            \item Method: Historical simulation
            \item Confidence level: 95\%
            \item Lookback period: 252 days
        \end{itemize}
    \item Expected Shortfall:
        \begin{itemize}
            \item Conditional mean of returns below VaR
            \item Same parameters as VaR calculation
        \end{itemize}
    \item Regime-specific metrics:
        \begin{itemize}
            \item Separate calculations for each regime
            \item Minimum 63 days of regime data required
        \end{itemize}
\end{itemize}

\subsection{Portfolio Construction}
\begin{itemize}
    \item Base weights:
        \begin{itemize}
            \item NVDA: 40\% (market cap weight)
            \item AMD: 30\% (market cap weight)
            \item INTC: 15\% (liquidity adjusted)
            \item ASML: 15\% (liquidity adjusted)
        \end{itemize}
    \item Position sizing rules:
        \begin{itemize}
            \item High volatility: 50\% of base weights
            \item Medium volatility: 100\% of base weights
            \item Low volatility: 125\% of base weights
        \end{itemize}
\end{itemize}

\subsection{Validation Metrics}
\begin{itemize}
    \item Model selection:
        \begin{itemize}
            \item AIC and BIC for HMM
            \item Validation loss for LSTM
            \item Regime persistence rates
        \end{itemize}
    \item Performance metrics:
        \begin{itemize}
            \item Sharpe ratio (annualized)
            \item Information ratio vs. PHLX Semiconductor Index
            \item Maximum drawdown by regime
        \end{itemize}
\end{itemize}
