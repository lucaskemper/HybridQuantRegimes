The semiconductor industry represents a critical component of modern technology infrastructure, characterized by high volatility, cyclical behavior, and complex market dynamics. Recent global events, including supply chain disruptions and geopolitical tensions, have highlighted the need for sophisticated risk management approaches in this sector.

This paper presents a hybrid framework combining Hidden Markov Models (HMM) and Long Short-Term Memory (LSTM) networks for detecting market regimes in semiconductor equities. The methodology integrates traditional statistical methods with deep learning techniques, featuring robust fallback mechanisms and multi-horizon analysis across short-term (21-day) and medium-term (63-day) market dynamics.

The empirical analysis examines daily return data from four major semiconductor companies (NVDA, AMD, INTC, ASML) from 2017-2023, with portfolio weights allocated based on market capitalization and liquidity (NVDA: 40\%, AMD: 30\%, INTC: 15\%, ASML: 15\%). The framework identifies three distinct market regimes: low volatility (56.62\% occurrence, mean return: 4.36\%, t=3.42, p < 0.05), medium volatility (31.89\% occurrence, mean return: 2.81\%, t=2.15, p < 0.05), and high volatility (11.48\% occurrence, mean return: -0.25\%, t=-0.18).

Initial implementation results from my proof-of-concept system demonstrate promising performance with a Sharpe ratio of 0.73 [0.65, 0.81] and Information ratio of 0.68 [0.61, 0.75]. The regime detection framework achieves persistence rates exceeding 80\% across all states, with rare direct transitions between extreme regimes. Portfolio risk metrics show strong statistical significance (p < 0.001), including Value at Risk (-3.99\% [-4.2\%, -3.7\%]) and Expected Shortfall (-5.74\% [-6.1\%, -5.4\%]). While these results suggest potential utility for dynamic risk management in semiconductor equity portfolios, further development is needed for production deployment.
