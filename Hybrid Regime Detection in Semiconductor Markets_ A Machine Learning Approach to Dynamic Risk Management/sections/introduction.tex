\section{Introduction}

\subsection{Market Context}
The semiconductor industry exhibits complex market dynamics characterized by distinct volatility regimes and rapid technological transitions. Recent global supply chain disruptions and geopolitical tensions have further amplified these challenges. My empirical analysis of four major semiconductor companies (NVDA, AMD, INTC, ASML) from 2017 to 2023 reveals significant regime-dependent behavior, with portfolio volatility reaching 42.72\% and Value at Risk (95\%) of -3.99\% \citep{alexander2008market}. These dynamics pose unique challenges for portfolio management and risk assessment \citep{mcneil2015quantitative}.

\subsection{Research Objectives}
This paper addresses three key challenges in semiconductor market analysis:

\begin{enumerate}
    \item \textbf{Regime Detection}: Development of a hybrid methodology combining Hidden Markov Models \citep{hamilton1989new} with LSTM networks \citep{fischer2018deep}, featuring robust fallback mechanisms and regime smoothing
    \item \textbf{Risk Assessment}: Implementation of regime-dependent risk metrics with dynamic position sizing and comprehensive stress testing \citep{mcneil2000estimation}
    \item \textbf{Portfolio Management}: Design of adaptive position sizing strategies with multi-level risk controls \citep{ang2002regime}
\end{enumerate}

\subsection{Methodological Framework}
The implementation introduces several key innovations:

\begin{enumerate}
    \item \textbf{Hybrid Detection System} \citep{hochreiter1997long}:
    \begin{itemize}
        \item Integration of HMM and LSTM with automated fallback mechanisms
        \item Mode-based regime smoothing with configurable windows
        \item Conservative regime selection in ensemble approach
    \end{itemize}

    \item \textbf{Multi-horizon Analysis} \citep{guidolin2011regime}:
    \begin{itemize}
        \item Short-term (21-day) volatility patterns
        \item Medium-term (63-day) market dynamics
        \item Comprehensive validation metrics including AIC and BIC
    \end{itemize}

    \item \textbf{Risk Management Framework} \citep{artzner1999coherent}:
    \begin{itemize}
        \item Dynamic VaR and Expected Shortfall calculations
        \item Volatility-scaled position sizing
        \item Multi-level risk controls (position, sector, portfolio)
    \end{itemize}
\end{enumerate}

\subsection{Key Findings}
The empirical analysis, based on 1,759 trading days of data, reveals three distinct market regimes with statistically significant characteristics:

\begin{itemize}
    \item Low Volatility Regime (56.62\% occurrence):
    \begin{itemize}
        \item Mean return: 4.36\% (t=3.42, p < 0.05)
        \item Volatility: 26.10\%
        \item High persistence (0.81 probability)
    \end{itemize}

    \item Medium Volatility Regime (31.89\% occurrence):
    \begin{itemize}
        \item Mean return: 2.81\% (t=2.15, p < 0.05)
        \item Volatility: 37.00\%
        \item Stable transitions (0.82 persistence)
    \end{itemize}

    \item High Volatility Regime (11.48\% occurrence):
    \begin{itemize}
        \item Mean return: -0.25\% (t=-0.18)
        \item Volatility: 57.06\%
        \item Clear regime boundaries
    \end{itemize}
\end{itemize}

\subsection{Paper Organization}
The remainder of this paper is organized as follows: Section II presents the methodological framework and implementation details. Section III describes the data and empirical validation approach. Section IV presents the results and statistical analysis. Section V discusses practical implications and limitations. Section VI concludes with directions for future research.
