\section{Methodology}

\subsection{Data Description}
I analyze daily return data from four major semiconductor companies (NVDA, AMD, INTC, ASML) from January 2017 to December 2023, comprising 1,759 trading days. The portfolio weights are allocated as follows: NVDA (40\%), AMD (30\%), INTC (15\%), and ASML (15\%). This weighting scheme reflects both market capitalization and liquidity considerations \citep{alexander2008market}.

\subsection{Hybrid Regime Detection Framework}

\subsubsection{Base Configuration}
The regime detection framework \citep{ang2002regime} is configured with the following parameters, as implemented in the codebase:
\begin{itemize}
    \item Number of regimes: 3 (Low, Medium, High Volatility)
    \item Window size: 21 trading days
    \item Minimum regime duration: 21 days
    \item Smoothing window: 5 days
    \item Feature set: returns, volatility, momentum
\end{itemize}

\subsubsection{Hidden Markov Model Component}
The HMM implementation \citep{hamilton1989new} serves as the foundation of the regime detection system, utilizing:
\begin{itemize}
    \item Volatility-based state identification
    \item Empirical quantile thresholds (33rd and 67th percentiles)
    \item Maximum likelihood estimation for transition probabilities
    \item Robust fallback mechanism for model convergence
\end{itemize}

\subsubsection{LSTM Enhancement}
The deep learning component \citep{hochreiter1997long, fischer2018deep} augments the base HMM with:
\begin{itemize}
    \item Sequential feature processing
    \item LSTM implementation with configurable parameters
    \item Standard optimization parameters
    \item Binary classification output
\end{itemize}

\subsubsection{Regime Smoothing}
The framework implements a smoothing mechanism to reduce spurious regime transitions:
\begin{equation}
R_t = \text{mode}(R_{t-w}, ..., R_t, ..., R_{t+w})
\end{equation}
where $w$ is the smoothing window size (default: 5 days).

\subsection{Risk Assessment Framework}

\subsubsection{Regime-Dependent Risk Metrics}
For each identified regime \citep{mcneil2000estimation}, I calculate:
\begin{itemize}
    \item Value at Risk (95\% confidence level)
    \item Expected Shortfall \citep{artzner1999coherent}
    \item Volatility using 21-day and 63-day windows
    \item Information ratios and regime-specific Sharpe ratios
\end{itemize}

\subsubsection{Position Sizing}
Position sizes are determined through a dynamic scaling approach:
\begin{equation}
P_t = \min(\max(P_b \cdot v_t, P_{min}), P_{max})
\end{equation}
where:
\begin{itemize}
    \item $P_b$ is the base position size
    \item $v_t$ is the volatility scaling factor
    \item $P_{min}, P_{max}$ are position limits
\end{itemize}

\subsection{Implementation Status}
The current implementation represents a proof-of-concept system developed in Python 3.10+:

\subsubsection{Core Components}
\begin{itemize}
    \item TensorFlow for basic LSTM implementation
    \item hmmlearn for HMM modeling
    \item pandas and numpy for data processing
    \item scikit-learn for preprocessing and validation
\end{itemize}

\subsubsection{Current Limitations}
\begin{itemize}
    \item Basic integration of HMM and LSTM methods
    \item Initial test coverage and validation framework
    \item Limited error handling capabilities
    \item Complex result structures requiring detailed knowledge
\end{itemize}

The implementation provides a foundation for future development while demonstrating the feasibility of the hybrid approach.
