\section{Discussion and Limitations}
Our results suggest that hybrid regime detection models integrating probabilistic and deep learning architectures can yield strong risk-adjusted performance in highly volatile sectors like semiconductors. The HMM offers interpretable, probabilistic regime segmentation, while the LSTM and Transformer architectures provide temporal modeling capabilities that capture nonlinear dependencies and dynamics across different market environments.

Notably, we have already implemented both LSTM and Transformer-based regime detection modules within the framework. The Transformer architecture, featuring positional encoding, multi-head self-attention, and residual connections, complements the LSTM's recurrent structure by offering parallelism and long-range feature extraction. This confirms the system’s extensibility and readiness for further advances in sequence modeling.

While HMMs performed strongly during this test period, the inclusion of deep learning components such as LSTMs and Transformers provides structural redundancy and model diversity. These architectures are particularly suited for detecting nonlinear regime shifts and complex temporal dependencies that may not have emerged in the recent semiconductor cycle. Their presence enables the framework to generalize beyond the sector and adapt to future environments with more fragmented or noisy regime transitions.

Second, the use of fixed technical features and hand-crafted indicators may limit generalization across broader asset classes. Incorporating multimodal inputs (e.g., news sentiment, earnings call transcripts, satellite data) would expand the model’s market understanding but comes with challenges in alignment, noise reduction, and overfitting.

Finally, the backtest period, although spanning multiple market cycles, is still relatively short and sector-specific. Broader testing on other sectors (e.g., energy, financials, healthcare) and in global markets (e.g., Taiwan, Korea, Europe) would help validate the robustness and generalizability of the framework.

\section{Practical Implications and Limitations}

\paragraph{Implications for Practitioners and Asset Managers}
The results of this study have several important implications for practitioners and asset managers seeking to implement systematic, regime-aware strategies in real-world portfolios:

\begin{itemize}
    \item \textbf{Interpreting the Results:} The achieved performance metrics---annualized return of 5.8\%, Sharpe ratio of 0.23, and max drawdown of 32\%---are realistic for a diversified, risk-managed equity sector strategy. These results demonstrate that regime-aware models can deliver robust, risk-adjusted returns even in volatile and uncertain market environments. The regime detection framework provides a transparent, interpretable mapping of market conditions, allowing practitioners to understand when the strategy is likely to perform well or face headwinds.
    \item \textbf{Real-World Use of the Regime Model:} The regime model enables dynamic adjustment of portfolio risk and exposure. For example, in low-volatility, positive-drift regimes, the model can recommend higher allocations or more aggressive position sizing, while in high-volatility or tail-risk regimes, it can automatically reduce risk, cut exposure, or move to defensive assets. By identifying regime shifts in real time, the model can serve as an early warning system for rising risk, allowing for timely de-risking, hedging, or rebalancing. Trading signals can be interpreted differently depending on the prevailing regime, improving the reliability of entry/exit decisions and reducing false positives during turbulent periods.
    \item \textbf{Limitations and Practical Considerations:} The backtest assumes sufficient liquidity to execute trades at daily closes. In practice, large orders or illiquid assets may incur slippage or market impact, especially during regime transitions or stress events. While the model includes transaction cost estimates, real-world costs may be higher, particularly for high-turnover strategies or in volatile markets. The approach is tested on a focused universe of semiconductor equities and ETFs; scaling to broader universes or higher-frequency data may require additional engineering, data infrastructure, and risk controls. Regime detection models, like all statistical models, are subject to estimation error, parameter instability, and potential overfitting. Out-of-sample validation and ongoing monitoring are essential to ensure continued robustness. The effectiveness of the regime and signal models depends on the quality and timeliness of input data; missing, stale, or erroneous data can degrade performance or trigger false regime shifts.
\end{itemize}

