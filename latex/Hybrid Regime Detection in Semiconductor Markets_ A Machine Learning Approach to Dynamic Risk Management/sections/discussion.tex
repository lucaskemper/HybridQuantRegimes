\section{Discussion and Limitations}
Our results suggest that hybrid regime detection models integrating probabilistic and deep learning architectures can yield strong risk-adjusted performance in highly volatile sectors like semiconductors. The HMM offers interpretable, probabilistic regime segmentation, while the LSTM and Transformer architectures provide temporal modeling capabilities that capture nonlinear dependencies and dynamics across different market environments.

Notably, we have already implemented both LSTM and Transformer-based regime detection modules within the framework. The Transformer architecture, featuring positional encoding, multi-head self-attention, and residual connections, complements the LSTM's recurrent structure by offering parallelism and long-range feature extraction. This confirms the system’s extensibility and readiness for further advances in sequence modeling.

However, several limitations remain. First, while both deep learning models and ensemble methods were used, performance gains were marginal in this particular backtest due to high regime persistence and strong agreement between models. More volatile or less stationary environments might highlight greater performance divergence.

Second, the use of fixed technical features and hand-crafted indicators may limit generalization across broader asset classes. Incorporating multimodal inputs (e.g., news sentiment, earnings call transcripts, satellite data) would expand the model’s market understanding but comes with challenges in alignment, noise reduction, and overfitting.

Finally, the backtest period, although spanning multiple market cycles, is still relatively short and sector-specific. Broader testing on other sectors (e.g., energy, financials, healthcare) and in global markets (e.g., Taiwan, Korea, Europe) would help validate the robustness and generalizability of the framework.

