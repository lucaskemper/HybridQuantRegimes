The semiconductor industry, a cornerstone of modern technology, is marked by rapid innovation, pronounced market swings, and sensitivity to macroeconomic and geopolitical events. These characteristics make risk management and portfolio allocation in semiconductor equities especially challenging. Traditional investment strategies often struggle to adapt to the frequent and abrupt regime shifts that typify this sector.

\textbf{Purpose and Motivation:} This work addresses the need for robust, adaptive tools to detect and respond to market regime shifts in semiconductor equities. Effective regime detection enables investors and risk managers to dynamically adjust strategies, mitigate drawdowns, and capitalize on emerging opportunities. In an environment where transitions between high-growth, correction, and crisis periods are common, timely and accurate regime identification is essential for both risk mitigation and performance enhancement.

\textbf{Why Regime Detection Matters in Semiconductors:} Semiconductor markets are uniquely exposed to global supply chain disruptions, technological breakthroughs, cyclical demand, and policy interventions. These factors can trigger rapid transitions between distinct market regimes, each with its own risk-return profile. Failure to recognize and adapt to these shifts can result in suboptimal performance or significant losses. Thus, advanced regime detection is not only of academic interest but also a practical necessity for investors in this space.

An overview of our hybrid HMM-LSTM regime detection methodology, which integrates probabilistic modeling and deep learning, is provided in the following section.
