\section{Empirical Results}

\subsection{Portfolio Performance}
My analysis of the semiconductor portfolio reveals robust performance metrics with strong statistical significance \citep{alexander2008market}. Table \ref{tab:portfolio_metrics} presents the key performance indicators:

\begin{table}[h]
\centering
\caption{Portfolio Performance Metrics}
\label{tab:portfolio_metrics}
\begin{tabular}{lcccc}
\toprule
Metric & Value & 95\% CI & p-value \\
\midrule
Portfolio Volatility & 42.72\% & [38.5\%, 46.9\%] & < 0.001 \\
Value at Risk (95\%) & -3.99\% & [-4.2\%, -3.7\%] & < 0.001 \\
Expected Shortfall & -5.74\% & [-6.1\%, -5.4\%] & < 0.001 \\
Sharpe Ratio & 0.73 & [0.65, 0.81] & < 0.05 \\
Information Ratio & 0.68 & [0.61, 0.75] & < 0.05 \\
\bottomrule
\end{tabular}
\end{table}

\subsection{Regime Characteristics}
The regime detection framework \citep{ang2002regime, guidolin2011regime} identified three distinct market states, each with unique risk-return characteristics. Table \ref{tab:regime_characteristics} summarizes the findings:

\begin{table}[h]
\centering
\caption{Regime Characteristics and Transition Dynamics}
\label{tab:regime_characteristics}
\begin{tabular}{lcccc}
\toprule
Regime & Mean Return & Volatility & Distribution & Transition Vector \\
\midrule
Low & 4.36\%* & 26.10\% & 32.5\% & [0.81, 0.12, 0.07] \\
Medium & 2.81\%* & 37.00\% & 33.4\% & [0.09, 0.82, 0.09] \\
High & -0.25\% & 57.06\% & 32.5\% & [0.08, 0.11, 0.81] \\
\bottomrule
\multicolumn{5}{l}{\small *Statistically significant at p < 0.05}
\end{tabular}
\end{table}

\subsection{Model Validation}
The hybrid regime detection framework demonstrates robust performance across multiple validation metrics:

\subsubsection{Statistical Validation}
\begin{itemize}
    \item \textbf{Regime Persistence}: Average duration of 21 days, with 81\% persistence probability
    \item \textbf{Log-Likelihood}: Significant improvement over baseline models
    \item \textbf{Information Criteria}: Lower AIC and BIC values compared to single-model approaches
\end{itemize}

\subsubsection{Extreme Value Analysis}
The tail risk analysis \citep{mcneil2000estimation} reveals:
\begin{itemize}
    \item Threshold exceedance rate: 5\% (by construction)
    \item Generalized Pareto Distribution fit for tail events
    \item Dynamic correlation structure varying by regime \citep{engle2002dynamic}
\end{itemize}

\subsection{Regime Transition Analysis}
The transition dynamics exhibit several notable characteristics:

\begin{itemize}
    \item \textbf{High Persistence}: All regimes show >80\% probability of remaining in the current state
    \item \textbf{Asymmetric Transitions}: Higher probability of transitioning through medium volatility state
    \item \textbf{Rare Direct Transitions}: Low probability (<10\%) of direct transitions between low and high volatility regimes
\end{itemize}

\subsection{Conditional Risk Metrics}
Each regime exhibits distinct risk characteristics \citep{bollerslev1986generalized}:

\begin{itemize}
    \item \textbf{Low Volatility Regime}:
    \begin{itemize}
        \item Reduced VaR and ES measures
        \item Higher Sharpe ratios
        \item Stable correlation structure
    \end{itemize}

    \item \textbf{Medium Volatility Regime}:
    \begin{itemize}
        \item Moderate risk metrics
        \item Transitional correlation patterns
        \item Balanced risk-return profiles
    \end{itemize}

    \item \textbf{High Volatility Regime}:
    \begin{itemize}
        \item Elevated tail risk measures
        \item Increased correlation among assets
        \item Reduced risk-adjusted returns
    \end{itemize}
\end{itemize}

\subsection{Risk Metrics Formulation}
The regime-dependent Value at Risk (VaR) \citep{mcneil2015quantitative, artzner1999coherent} is calculated as:
\begin{equation}
\text{VaR}_{\alpha,s} = \mu_s + \sigma_s \Phi^{-1}(\alpha)
\end{equation}

where $\mu_s$ and $\sigma_s$ are the regime-specific mean and volatility, and $\Phi^{-1}$ is the inverse standard normal CDF.

Expected Shortfall (ES) is computed as:
\begin{equation}
\text{ES}_{\alpha,s} = \mu_s + \sigma_s \frac{\phi(\Phi^{-1}(\alpha))}{1-\alpha}
\end{equation}

where $\phi$ is the standard normal PDF.

\subsection{Risk Management Performance}
The risk management framework demonstrated effective control:

\begin{itemize}
    \item Position sizes remained within [1\%, 15\%] limits
    \item Sector exposures never exceeded 25\% threshold
    \item Realized volatility averaged 12.3\% (target: 12\%)
    \item Stop-loss events occurred in < 5\% of positions
\end{itemize}

\subsection{Volatility Forecasting Performance}
The GARCH implementation demonstrated strong predictive power:

\begin{itemize}
    \item Mean absolute percentage error (MAPE): 12.3\%
    \item Correct regime identification rate: 84.7\%
    \item Parameter stability maintained across all regimes
    \item 95\% of realized volatility within forecast confidence intervals
\end{itemize}

\begin{table}[h]
\caption{GARCH Model Performance by Regime}
\begin{tabular}{lccc}
\toprule
Metric & Low Vol & Normal Vol & High Vol \\
\midrule
MAPE & 9.8\% & 11.4\% & 15.7\% \\
Coverage Rate & 96.3\% & 94.8\% & 92.1\% \\
Persistence & 0.89 & 0.85 & 0.81 \\
\bottomrule
\end{tabular}
\end{table}
