\section{Results and Empirical Evaluation}

\subsection{5.1 Experimental Setup}

We evaluate the proposed hybrid regime detection and dynamic risk management system on a portfolio of semiconductor equities and ETFs: \texttt{SMH}, \texttt{SOXX}, \texttt{NVDA}, \texttt{AMD}, \texttt{TSM}, \texttt{INTC}, \texttt{QCOM}, and \texttt{AVGO}. The backtest spans January 1, 2019 to January 1, 2024, using daily adjusted closing prices.

A walk-forward validation procedure is employed to prevent data leakage and ensure realistic evaluation. Transaction costs of 5 basis points and fixed slippage are applied. Daily rebalancing is used, though slower rebalancing frequencies were tested and yielded similar outcomes.

Performance metrics include annualized return, volatility, Sharpe ratio, Calmar ratio, maximum drawdown, win rate, and turnover. We also evaluate regime persistence, regime separation, and transition confidence.

\subsection{5.2 Regime Characteristics}

The hybrid model identifies $K=5$ latent regimes, selected based on out-of-sample log-likelihood and BIC. These regimes exhibit distinct profiles in terms of volatility, return drift, and macro-feature dependencies. Table~\ref{tab:regimes} reports the time spent in each regime.

\begin{table}[h]
\centering
\begin{tabular}{lcc}
\toprule
\textbf{Regime} & \textbf{Observation Count} & \textbf{Proportion of Time (\%)} \\
\midrule
Regime 2 (low-vol, steady drift)   & 411 & 32.4\% \\
Regime 4 (moderate vol, uptrend)   & 270 & 21.3\% \\
Regime 0 (neutral, low momentum)   & 218 & 17.2\% \\
Regime 1 (volatile, correction)    & 180 & 14.2\% \\
Regime 3 (high-vol, tail risk)     & 178 & 14.0\% \\
\bottomrule
\end{tabular}
\caption{Distribution of Regime Observations (2019–2024)}
\label{tab:regimes}
\end{table}

The model demonstrates high regime persistence (94.6\%) and minimal churning, supported by a strongly diagonal transition matrix (Appendix~A). Regime overlays on price series show strong alignment with key market events, including the COVID-19 crash and the 2022 semiconductor correction.

\subsection{5.3 Strategy Performance}

The hybrid strategy achieves robust performance across both absolute and risk-adjusted metrics. Annualized return exceeds 5.4\% with a Sharpe ratio of 1.34 and annualized volatility below 2.6\%. Drawdowns are tightly controlled, with a maximum drawdown of only 1.35\%. The Calmar ratio exceeds 4.0, suggesting strong downside protection and efficient capital growth under volatility-aware risk management.

Daily turnover remains modest (1.17\%), indicating practical implementation feasibility. Win rate exceeds 5\%, reflecting consistent signal effectiveness despite low average volatility in the strategy’s holdings.

\subsection{5.4 Risk and Regime Analysis}

Figure~\ref{fig:equity_curve_overlay} shows the strategy’s equity curve with regime overlays. Regime transitions are sparse and informative, coinciding with notable market events. Risk-return scatterplots (Appendix~B) demonstrate clear clustering by regime, with Regime 2 offering the most attractive profile (low volatility, positive drift) and Regime 3 capturing extreme downside behavior.

Rolling Sharpe ratios remain stable, with the strategy maintaining performance even during high-volatility episodes. Confusion matrices confirm consistent regime classification over time, with no evidence of regime flipping or instability.
